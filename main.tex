% ! TeX root = main.tex
\documentclass[aspectratio=169]{beamer}
\usepackage[utf8]{inputenc}
\usepackage{multicol}
\usepackage{pgfpages}
\usepackage[draft]{pdfcomment}
\title{Research directions for Aggregate Computing with
Machine Learning}
\author[G.Aguzzi]{
  \textbf{Gianluca Aguzzi}\inst{1}
}
\institute{
  \inst{1}
  \texttt{Alma Mater Studiorum} -- Università di Bologna, Cesena, Italy
}
\usetheme{material}
%& Theme
\usetheme{material}
\useLightTheme
\usePrimaryBlueGrey
\useAccentIndigo
%% Document starts
\begin{document}
\begin{frame}
  \titlepage
  \pdfcomment{
    Here I think I do not say so much, just a little bit of an introduction about who I am
  }
\end{frame}
%%%%%%%%%%%%%%%%%%%%%%%%%%%%%%%%%%%%%%%%%%%%%%%%%%%%%%%%%%%%%%%%%%%%%%%%%%%%%
%%%%%%%%%%%%%%%%%%%%%%%%%%%%%%%%%%%%%%%%%%%%%%%%%%%%%%%%%%%%%%%%%%%%%%%%%%%%%
%%%%%%%%%%%%%%%%%%%%%%%%%%%%%%%%%%%%%%%%%%%%%%%%%%%%%%%%%%%%%%%%%%%%%%%%%%%%%
\section{Collective (Self) Adaptive System}
\begin{frame}{Background}
  \pdfcomment{
    This kind of system is the target of our approach, they consist in a large number of 
    intercomunicating entities that expose some kind of intelligent behaviour distilled via learning.
    Learning is a key mechanism to enforce adaptivity against faiulure, enviromental changes, openess and eteregeity
    Over the years, in literature a plethora of approach aim at managin this complexity. In my research group we are focused on 
    Aggregate Computing.
  }
\end{frame}
%%%%%%%%%%%%%%%%%%%%%%%%%%%%%%%%%%%%%%%%%%%%%%%%%%%%%%%%%%%%%%%%%%%%%%%%%%%%%
%%%%%%%%%%%%%%%%%%%%%%%%%%%%%%%%%%%%%%%%%%%%%%%%%%%%%%%%%%%%%%%%%%%%%%%%%%%%%
%%%%%%%%%%%%%%%%%%%%%%%%%%%%%%%%%%%%%%%%%%%%%%%%%%%%%%%%%%%%%%%%%%%%%%%%%%%%%
\section{State of the art solution}
\begin{frame}{Aggregate Computing}
  \pdfcomment{
    It is a top-down approach by which it is possibile to coordinate large scale, possibly etherogenous, very dinamyc 
    system. It mainly consist in a manipulation of a distributed data structure called Computational Field. Instead of 
    swarm intelligent approach, the aim here is to "program" self-organisation, posing it as something like first-class citezen 
    of the language. The program then can be splitted in each node, so collective behaviour can easly scale up In
    very large node system (hunder or thunsand nodes).
    The key insight of this apporach is their compositionality inspired by functional programming.
    Indeed, the paradigm is based on a little algebra called field-calculus, that express the minimal constructs
    able to express whatever spatio-temporal computation.
    Over it, there where build a middle-level abstraction so-called Building-Block, consisting in usual pattern
    used in Aggregate Computing.
    With them then, we built application level API to support developers in making complex large scale application.
    Over last ten/fiveteen years, this metholodigy growth rapdly. It is applied in many context like
    crowd engineering, swarm robotics, and smart cities.
    Pragramtically, we see that it is not easy to build building block that works in various scenario (e.g. high node mobility).
    Sometimes it is a very tricky and fine tuning process by wich we have to put some magic constant in order to make our application 
    works
  }
\end{frame}
%%%%%%%%%%%%%%%%%%%%%%%%%%%%%%%%%%%%%%%%%%%%%%%%%%%%%%%%%%%%%%%%%%%%%%%%%%%%%
%%%%%%%%%%%%%%%%%%%%%%%%%%%%%%%%%%%%%%%%%%%%%%%%%%%%%%%%%%%%%%%%%%%%%%%%%%%%%
%%%%%%%%%%%%%%%%%%%%%%%%%%%%%%%%%%%%%%%%%%%%%%%%%%%%%%%%%%%%%%%%%%%%%%%%%%%%%
\section{Machine Learning}
\begin{frame}{Machine Learning}
  \pdfcomment {
    Another way to handle these system make them learn the right behaviour. Here, thanks also
    to the new waves of interest in Machine Learning, there are a lot of insight, solutions and ideas.
    Reinforcement Learning, Superverised Learning, Evolutionary computing are all of the 
    tatics that is used in research. Nevertheless they goodness, the solution are in general
    very specific and cannot be used easly in different scenario. Futhermore, some thereom proven in single agent
    context cannot be extended in a multi agent setting.
    Here in fact, we have to cocern about non-stationary environment, very large scale space, distributed control
    and massive large scale that make ML approach very hardly to scale up in diffent contexts.
  }
\end{frame}
%%%%%%%%%%%%%%%%%%%%%%%%%%%%%%%%%%%%%%%%%%%%%%%%%%%%%%%%%%%%%%%%%%%%%%%%%%%%%
%%%%%%%%%%%%%%%%%%%%%%%%%%%%%%%%%%%%%%%%%%%%%%%%%%%%%%%%%%%%%%%%%%%%%%%%%%%%%
%%%%%%%%%%%%%%%%%%%%%%%%%%%%%%%%%%%%%%%%%%%%%%%%%%%%%%%%%%%%%%%%%%%%%%%%%%%%%
\section{Problem statement}
\begin{frame}{Problem statement}
  \pdfcomment{
    So mainly our overall goal is:
    Extended Common aggregate computing paradign with Machine Learning capability in order to improve adaptivity and to simplify collective behaviour definition
  }
\end{frame}
%%%%%%%%%%%%%%%%%%%%%%%%%%%%%%%%%%%%%%%%%%%%%%%%%%%%%%%%%%%%%%%%%%%%%%%%%%%%%
%%%%%%%%%%%%%%%%%%%%%%%%%%%%%%%%%%%%%%%%%%%%%%%%%%%%%%%%%%%%%%%%%%%%%%%%%%%%%
%%%%%%%%%%%%%%%%%%%%%%%%%%%%%%%%%%%%%%%%%%%%%%%%%%%%%%%%%%%%%%%%%%%%%%%%%%%%%
\section{Motivation}
\begin{frame}{Motivation}
  \pdfcomment{
    But why? In some sense with this combination we want to improve Aggregate Computing.
    Indeed currently it have a key ingredient to make it excellent in applying in CSAS, take to
    its scale indipendent nature.
    Futhermore we want to explore a new way to declare behaviour. Where a part of collective behaviour
    are still expressed in term of API composition and call, and a part that is hard to be described is 
    distilled by learning
    Finally we want to navigate deeply the Machine Learning approach in this very hard kind of system. Even if in the last 
    year some work initially to be done, Field-Based coordination have never meet ML. It might open new opportunities and technique.  
  }
\end{frame}
%%%%%%%%%%%%%%%%%%%%%%%%%%%%%%%%%%%%%%%%%%%%%%%%%%%%%%%%%%%%%%%%%%%%%%%%%%%%%
%%%%%%%%%%%%%%%%%%%%%%%%%%%%%%%%%%%%%%%%%%%%%%%%%%%%%%%%%%%%%%%%%%%%%%%%%%%%%
%%%%%%%%%%%%%%%%%%%%%%%%%%%%%%%%%%%%%%%%%%%%%%%%%%%%%%%%%%%%%%%%%%%%%%%%%%%%%
\section{Research Question}
\begin{frame}{Research Questions}
  \pdfcomment 
  {
    The research direction that we aim to tackle are:
    i) what kind of machine learning approach make sense in combination to Aggregate Computing?
    ii) at what level of abstraction ML can be useful for Aggregate Computing?
    iii) what Aggregate Computing have in common with ML applied to Collective Adaptive System?
    This what guiding us in my first year of research.
  }
\end{frame}
%%%%%%%%%%%%%%%%%%%%%%%%%%%%%%%%%%%%%%%%%%%%%%%%%%%%%%%%%%%%%%%%%%%%%%%%%%%%%
%%%%%%%%%%%%%%%%%%%%%%%%%%%%%%%%%%%%%%%%%%%%%%%%%%%%%%%%%%%%%%%%%%%%%%%%%%%%%
%%%%%%%%%%%%%%%%%%%%%%%%%%%%%%%%%%%%%%%%%%%%%%%%%%%%%%%%%%%%%%%%%%%%%%%%%%%%%
\section{Early results}
\begin{frame}{Early results}
  \pdfcomment {
    The main direction taken currently are:
    i) Combine RL with Supervised Learning
    ii) Combine Aggregate Computing with Evolutionary Computing
    iii) And the finally combine Aggregate Computing with Reinforcement Learning
}
\end{frame}
%%%%%%%%%%%%%%%%%%%%%%%%%%%%%%%%%%%%%%%%%%%%%%%%%%%%%%%%%%%%%%%%%%%%%%%%%%%%%
%%%%%%%%%%%%%%%%%%%%%%%%%%%%%%%%%%%%%%%%%%%%%%%%%%%%%%%%%%%%%%%%%%%%%%%%%%%%%
%%%%%%%%%%%%%%%%%%%%%%%%%%%%%%%%%%%%%%%%%%%%%%%%%%%%%%%%%%%%%%%%%%%%%%%%%%%%%
\section{Enforcements}
\begin{frame}{Enforcements}
  \pdfcomment{
    Currently we setting our learning problem as an Homegenous Team Learning, that is some sense is 
    very near to Aggregate Computing Setting. Indeed, each agent should execute the same behaviour in order
    to reach a global result.
    Then, we perform this test offline, using a Alchemist, a well-know and robust meta simulator used with
    Aggregate Computing.
    Then, we are focused on the Buildign Block improvement and definition level, because it is something very important
    in our research area. Indeed, if we can boost up block defintion or improvement, it will gain a boost in our 
    direction.
  }
\end{frame}
%%%%%%%%%%%%%%%%%%%%%%%%%%%%%%%%%%%%%%%%%%%%%%%%%%%%%%%%%%%%%%%%%%%%%%%%%%%%%
%%%%%%%%%%%%%%%%%%%%%%%%%%%%%%%%%%%%%%%%%%%%%%%%%%%%%%%%%%%%%%%%%%%%%%%%%%%%%
%%%%%%%%%%%%%%%%%%%%%%%%%%%%%%%%%%%%%%%%%%%%%%%%%%%%%%%%%%%%%%%%%%%%%%%%%%%%%
\section{Computational Model}
\begin{frame}{Interaction model}
  \pdfcomment{
    In aggregate computing, we see a systema as a node collected with each other by some
    neighbourh relatioship that could be phisical (namely a node have a link with another node)
    or logical/position based.
    Each node have only this partial vision of the entire aggregate. The interaction with
    the neighbourhood happen via message passing. The information exechanged depend on the
    execution model of this system.
  }
\end{frame}
%%%%%%%%%%%%%%%%%%%%%%%%%%%%%%%%%%%%%%%%%%%%%%%%%%%%%%%%%%%%%%%%%%%%%%%%%%%%%
%%%%%%%%%%%%%%%%%%%%%%%%%%%%%%%%%%%%%%%%%%%%%%%%%%%%%%%%%%%%%%%%%%%%%%%%%%%%%
%%%%%%%%%%%%%%%%%%%%%%%%%%%%%%%%%%%%%%%%%%%%%%%%%%%%%%%%%%%%%%%%%%%%%%%%%%%%%
\begin{frame}{Execution model}
  \pdfcomment{
    An aggregate program has two point of view. A collective one that speak in term of
    computional field evaluation and a local one that follow the global behaviour executing
    periodically a round, i.e the atomic part of a computation in an aggregate system.
    A round is comprised by: a context creation: by which node take values from sensors, message passed
    from neighbourhood and the old value produce. Then with it, evaluate the aggregate program
    producing an Export (namely a data structure contained all the information necessary to neighbours to evaluate the 
    programm), with this export a node can perform some local actuation and then share it with 
    the entire neighbourhood.
    Note that each computation is completly asichrnours w.rt. other node.
  }
\end{frame}
%%%%%%%%%%%%%%%%%%%%%%%%%%%%%%%%%%%%%%%%%%%%%%%%%%%%%%%%%%%%%%%%%%%%%%%%%%%%%
%%%%%%%%%%%%%%%%%%%%%%%%%%%%%%%%%%%%%%%%%%%%%%%%%%%%%%%%%%%%%%%%%%%%%%%%%%%%%
%%%%%%%%%%%%%%%%%%%%%%%%%%%%%%%%%%%%%%%%%%%%%%%%%%%%%%%%%%%%%%%%%%%%%%%%%%%%%
\begin{frame}{Gradient/Hop count example}
  \pdfcomment{
    Currently, our reference building block is the gradient from a source zone.
    It produces a computational field in which each node produce the distance from a 
    node marked as source. Hop count is very similar but instead of distance the program
    return the number of hop from a target node.
    A naive implemetation of this algorithms consist in take the minimum value from the neighbourhood Field
    incresed with the distance from it.
    Put an image that shows graphically the meaning of gradient and hop count. 
  }
%% Here put an image that show the step to compute this field.
\end{frame}
\begin{frame}{Gradient/Hop count example: Problem}
  \pdfcomment{
    This very simple solution soffer of slow rising problem: Suppose you have two source and one of them
    at some point in time disappear. In this part node need more time to reach the right value
    in difference to the initial part.
    So, our first effort consist in increase the converge time of basic hop count algorithm. This is already done
    manually, but we what something that learn alone how to bring to a faster convergence.
  }
\end{frame}
%%%%%%%%%%%%%%%%%%%%%%%%%%%%%%%%%%%%%%%%%%%%%%%%%%%%%%%%%%%%%%%%%%%%%%%%%%%%%
%%%%%%%%%%%%%%%%%%%%%%%%%%%%%%%%%%%%%%%%%%%%%%%%%%%%%%%%%%%%%%%%%%%%%%%%%%%%%
%%%%%%%%%%%%%%%%%%%%%%%%%%%%%%%%%%%%%%%%%%%%%%%%%%%%%%%%%%%%%%%%%%%%%%%%%%%%%
\begin{frame}{Aggregate Computing and Supervised Learning}
  \pdfcomment{In this kind of setting, we assume that we know the right result of the collective at certain point in time.
    So, starting to initial global setting A, we would reach a stable position B. In the gradient problem for example, the stable position
    consist in the distance from a node.
    In some way it is inspired by
    Graph Neural Network, in which the input of the net is a Graph and the ground truth again could be a graph.
    here there are some experiments done in this direction. What we observe it that is it very hard to gain perfomance 
    in this way. Indeed we does not know the right result for each time step, making hard to guide the collective in a performance boost.
  }
\end{frame}
%%%%%%%%%%%%%%%%%%%%%%%%%%%%%%%%%%%%%%%%%%%%%%%%%%%%%%%%%%%%%%%%%%%%%%%%%%%%%
%%%%%%%%%%%%%%%%%%%%%%%%%%%%%%%%%%%%%%%%%%%%%%%%%%%%%%%%%%%%%%%%%%%%%%%%%%%%%
%%%%%%%%%%%%%%%%%%%%%%%%%%%%%%%%%%%%%%%%%%%%%%%%%%%%%%%%%%%%%%%%%%%%%%%%%%%%%
\begin{frame}{Aggregate Computing and Evolutionary computing}
  \pdfcomment{
    Here we do not need a right ground truth, but it is necessary only to specify a fitness fuction that
    evaluate the overall behaviour of the system. This kind of method is used in swarm robotics and even in 
    automatic design. Here we observe very low performance in converging to a good solution. Sometimes it never reach one and
    locked the search phases into local minimum
  }
\end{frame}
%%%%%%%%%%%%%%%%%%%%%%%%%%%%%%%%%%%%%%%%%%%%%%%%%%%%%%%%%%%%%%%%%%%%%%%%%%%%%
%%%%%%%%%%%%%%%%%%%%%%%%%%%%%%%%%%%%%%%%%%%%%%%%%%%%%%%%%%%%%%%%%%%%%%%%%%%%%
%%%%%%%%%%%%%%%%%%%%%%%%%%%%%%%%%%%%%%%%%%%%%%%%%%%%%%%%%%%%%%%%%%%%%%%%%%%%%
\section{References}
\begin{frame}{Aggregate Computing and Reinforcement Learning}
  \pdfcomment{
    This combination seems to be one of the most appropiate for several reason:
    i) in literature (cite) seems to be the best way to integrate learning in CSAS
    ii) it is built to maximise the long term result and do not maximise only local best response.
    iii) even if aggregate program seems to be a distributed regression, it is quite easy to decode problems in term of agents state and actions.
  }
\end{frame}
\begin{frame}{References}

\end{frame}
\end{document}
