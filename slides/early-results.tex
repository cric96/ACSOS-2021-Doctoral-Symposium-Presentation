% ! TeX root = ...

\begin{frame}{Early results}
  \begin{multicols}{2}
    \begin{card}[Setting]
      \begin{itemize}
        \item <1->Focus on simple but well-known problems in Aggregate Computing
        \item <2->Learning exploited to guide building-block improvements
        \item <3->Verifying what kind of approach is well-suited for Aggregate Computing
      \end{itemize}
      \pdfcomment{
        Indeed we obtained some interesting early results, 
        we mainly try to create a very simple scenario to demonstrate
        that learning actually improve aggregate computing. So, for this reason, 
        we focus on refining the building blocks so that we can improve our future and current solutions.
        For having a clear vision of various approaches, we initially try to explore each solution that 
        consists of: i) Combine Aggregate Computing with Supervised Learning ii)
        Combine Aggregate Computing with Evolutionary Computing 
        iii) Combine Aggregate Computing with Reinforcement Learning. 
      }
    \end{card}
    \pause[4]
    \begin{cardRed}[\textbf{Constraints}]
      \begin{itemize}
        \item <4->Learning problem framed as Homogenous Team Learning~\cite{DBLP:journals/aamas/PanaitL05}
        \item <5->Learning performed off-line~\cite{alchemist-jos2013}
      \end{itemize}
      \pdfcomment{
        We formulate our learning problem as Homogenous Team Learning, 
        which is very close to the Aggregate Computing setting. 
        Indeed, each agent should perform the same behaviour to obtain a collective result. 
        We then run this test offline using Alchemist, a well-known and robust simulator 
        used with Aggregate Computing.  
      }
    \end{cardRed}
    
  \end{multicols}
\end{frame}