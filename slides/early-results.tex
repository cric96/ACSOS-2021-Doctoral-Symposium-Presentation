% ! TeX root = main.tex

\begin{frame}{Early results}
  \begin{multicols}{2}
    \begin{card}[Setting]
      \begin{itemize}
        \item <1->Focus on simple but well-know problem in Aggregate Computing
        \item <2->Learning used to guide building-block improvement
        \item <3->Veryfying what kind of approach is well-suited for Aggregate Computing
      \end{itemize}
      \pdfcomment{
        So, speaking about the early results, we mainly try to create a very simple 
        scenario by which learning has a sense at all.
        So, for this reason, we focus on improving the building blocks because this is something very important in our research area. 
        If we can improve the building block definition process, it then will improve our future and current solutions.  
        For having a clear vision of various approaches, we initially try to explore each solution
        that consists in: i) Combine Aggregate Computing with Supervised Learning 
        ii) Combine Aggregate Computing with Evolutionary Computing iii) 
        iii) Combine Aggregate Computing with Reinforcement Learning.
      }
    \end{card}
    \pause[4]
    \begin{cardRed}[\textbf{Constraints}]
      \begin{itemize}
        \item <4->Learning problem framed as Homogenous Team Learning
        \item <5->Learning done off-line
      \end{itemize}
      \pdfcomment{
        We formulate our learning problem as Homogenous Team Learning, 
        which is very close to the Aggregate Computing setting. 
        Indeed, each agent should perform the same behaviour to obtain a collective result.
        We then run this test offline using Alchemist, 
        a well-known and robust simulator used with Aggregate Computing. 
      }
    \end{cardRed}
    
  \end{multicols}
\end{frame}