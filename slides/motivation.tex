% ! TeX root = ...

\begin{frame}{Motivation}
  \begin{backgroundblock} 
    \includegraphics[width=\paperwidth]{img/motivation.jpg} 
  \end{backgroundblock} 
  \begin{card}
    \begin{itemize}
      \item[\faCheck]  <1-> Aggregate Computing is scale independent by construction
      \item[\faCheck]  <2-> Hybrid collective program description
      \item[\faCheck]  <3-> Try to improve current state-of-the-art Machine Learning applying in CSAS
    \end{itemize}
  \end{card}
  \pdfcomment{
    Indeed, aggregate computing has a key component that makes 
    it eminently suitable for application in CSAS, namely its scale-independent nature. 
    A program is expressed for a logical system and theoretically can be transparently
    applied to any system size (or to any topology, thanks to the recent work of the pulverised architecture). 
    Moreover, we would like to explore a new way to declare collective behaviour: 
    A part of the collective behaviour is still expressed in terms of API composition and function calls, 
    while another part, where humans have difficulty expressing the correct behaviour, is distilled through 
    learning. Finally, we want to deepen the approach of Machine Learning in this very complex kind of 
    system. Although some work has been started in the last years, Aggregate Computing has never meet Machine Learning. 
    It could open new opportunities and techniques that can also improve the current results of state-of-the-art machine learning algorithms. 
  }
\end{frame}