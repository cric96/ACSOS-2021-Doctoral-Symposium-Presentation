% ! TeX root = ...

\begin{frame}{Gradient/Hop count example}
  \begin{cardRed}[\textbf{Problem} \faThumbsDown]
    Naive solutions suffers of the slow-rising problem.
  \end{cardRed}
  \begin{multicols}{3}
    \cardImg{img/slow-rising-problem}{0.31\textwidth}
    \presentationGraphics{img/slow-rising-problem-1}{1}{0.31}
    \presentationGraphics{img/slow-rising-problem-2}{1-2}{0.31}
  \end{multicols}
  \pause[3]
  \pdfcomment{
    This solution suffers from the slow-rising problem: 
    suppose you have two sources and one of them eventually disappears. 
    In this case, the nodes take more time to reach the correct value, 
    as opposed to the time taken at the beginning of the algorithm's execution.
    So our first attempt is to reduce the convergence time of the basic hop-count algorithm. 
    This is already done algorithmically (put citations), but we need something that learns how to achieve faster 
    convergence on its own.  
  }
\end{frame}