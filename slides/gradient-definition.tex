\begin{frame}{Gradient/Hop count example}
  \begin{multicols}{2}
    \only<1>{
      \cardImg{img/gradient}{0.5\textwidth}
    }
    \only<2>{
      \cardImg{img/gradient-1}{0.5\textwidth}
    }
    \only<3>{
      \cardImg{img/gradient-2}{0.5\textwidth}
    }
    \begin{card}[Definition]
      A program that produce a computational 
      field where each node
      contains the distance from a source zone.
    \end{card}
  \end{multicols}
  \pdfcomment{
    Currently, our reference building block is the gradient from a source zone.
    It generates a computational field in which each node generates the distance from a node marked as a source. 
    The hop count problem is similar. Instead of evaluating the distance, the program returns the number of hops from a destination node.
    A naive implementation of this algorithm consists in: i) taking the minimum value from 
    the neighbour field and ii) incremented by the distance to it. eventually, the right field 
    will be produced.
  }
%% Here put an image that show the step to compute this field.
\end{frame}