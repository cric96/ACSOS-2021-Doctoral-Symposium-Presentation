\begin{frame}{Computational model}
  \begin{multicols}{2}
    \begin{card}
      \begin{itemize}
        \item <1->Ensemble of node with an identifier
        \item <2->Each node has a local-view (i.e. neighbours relationship)
        \item <3->Interaction happens with message passing (done continously, see next slide).
      \end{itemize}    
    \end{card}
    \centering
    \cardImg{img/interaction-msg}{0.5\textwidth}
  \end{multicols}
  \pdfcomment{
    Before I talk about my early works, let me briefly explain what aggregate computing is. 
    Here, we view a system as a large group of computing nodes connected by a neighbour relationship, 
    which can be either physical (i.e., one node has a connection to another node) or logical/position-based.
    Each node has only a partial view of the entire system. Interaction with the neighbourhood 
    is done through message passing.
  }
\end{frame}