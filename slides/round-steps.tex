% ! TeX root = ...

\begin{frame}{Computational model}
  \begin{multicols}{2}
    \begin{card}[Round steps]
      \begin{enumerate}
        \item <1-> context creation
        \item <2-> program evaluation producing an export
        \item <3-> export sharing to neighbourhood
      \end{enumerate}
    \end{card}
    \only<1>{
      \cardImg{img/context}{0.5\textwidth}
    }
    \only<2>{
      \cardImg{img/interaction-execution}{0.5\textwidth}
    }
    \only<3>{
      \cardImg{img/interaction-exchange}{0.5\textwidth}
    }
  \end{multicols}
  \pdfcomment{
    In these messages, the nodes share the local knowledge necessary to run a program. Indeed
    an aggregate program has two points of view: i) A collective viewpoint, which refers to the evaluation of computational fields, and a local viewpoint, which follows the global behaviour and periodically executes a round, i.e., 
    the atomic part of the computation in an aggregate system.
    A round consists of: i) a context creation where nodes take in values from sensors, messages (also called export) from the neighbourhood, and their old output. 
    ii) Then, the aggregate program is evaluated, producing the next export that contains the current program output. This data can be also used to perform local actuations.
    iii) finally, each node shares the export in broadcast to the entire neighbourhood.
    Note that any computation is asynchronous to other nodes.
  }
\end{frame}